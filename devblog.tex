\documentclass{article}
\usepackage{graphicx}
\usepackage{circuitikz}
\usepackage{amsmath}
\usepackage{float}
\usepackage{lewis}
\usepackage{tikz}
\usetikzlibrary{arrows,snakes,backgrounds,shapes}

\title{Avocat Dev-Blog}
\author{Timothy Player}

\begin{document}
\maketitle

\section{Sept. 1st, 2021: Timothy Player Server Time!}

I begin by stating the simple architecture. Let there be a user Bob that has no clue how to Google so he uses Avocat to Google and debug for him.
He gets help from a Daemon, what she does is communicate with a server. The server sends back a response as a snippet of 
code to be executed to resolve the error. A simplified problem with many sub problems. First how does the server get the solutions? Google? A curated list of 
solutions? Trial and error with its previous solutions? We for now Google is our best bet. Below is a simplified 

\begin{figure}[H]
    \begin{tikzpicture}
    \draw (2,5) node (server) {Server};
    \draw (3,3) node (listener) {Listener};
    \draw (1,3) node (parser) {Parser};
    \draw (3,1) node (resolver) {Resolver};
    \draw (-1,5) node[cloud, draw,cloud puffs=10,cloud puff arc=120, aspect=2, inner ysep=1em] (google) {Google};
    \draw (0,0) -- (4,0) -- (4,4) -- (0,4) -- (0,0);
    \draw (5,5) -- (5,0);
    \draw (6,0) -- (10,0) -- (10,4) -- (6,4) -- (6,0);
    \draw (8,5) node {Client};
    \draw (7,3) node (daemon) {Daemon};
    \draw[->] (daemon.west) -- (listener.east);
    \draw[->] (listener.west) -- (parser.east);
    \draw[->] (parser.west) .. controls +(left:5mm)  .. (google.south);
    \draw[->] (google.south)  .. (resolver.west);
    \draw[->] (resolver.east) .. controls +(right:5mm)  .. (daemon.south);
    \end{tikzpicture}
    \caption{Server/Client Architecture}
\end{figure}

\end{document}